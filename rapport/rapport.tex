\documentclass[12pt]{article}

% Chargement des packages nécessaires
\usepackage[utf8]{inputenc}
\usepackage[T1]{fontenc}
\usepackage[francais]{babel}
\usepackage{graphicx}


% Définition de la page de garde avec titre, auteurs, date

\title{Architecture des Ordinateurs : Rapport de Projet}

\author{Candice Bentéjac et Paul Beziau\\
  Licence 2 d'Informatique\\ 
  Architecture des Ordinateurs}

\date{\today}

% Début du document
\begin{document}

% Chargement page de garde
\begin{figure}
  \includegraphics{logo_ub.png}
  \end{figure}
\maketitle

% Permet de sauter 5 lignes
\vspace{5\baselineskip}
% Résumé du rapport
e \abstract{Ce rapport tient lieu de compte-rendu du binôme composé de Paul Beziau et Candice Bentéjac (groupe IN400A2) dans le cadre du projet d'architecture des ordinateurs (UE Architecture des Ordinateurs).}

% Nouvelle page
\newpage

% Chargement table des matières
\tableofcontents
\newpage

\section*{Introduction}
\paragraph{}Dans le cadre de l'Unité d'Enseignement ``Architecture des Ordinateurs'' (INF 155), il nous a été demandé de réaliser un projet consistant à modifier les fichiers sources des simulateurs d'architecture Y86, et ce tant sur la version {\itshape séquentielle} que sur la version {\itshape pipelinée}. Ce projet était constitué de trois parties, présentées sous forme d'exercices.

\paragraph{}Le premier exercice consistait à libérer des emplacements parmi les {\itshape opcodes} Y86. Il nous fallait donc factoriser des instructions peu ou prou semblables afin de pouvoir les utiliser avec un seul {\itshape opcode} plutôt que plusieurs.

\paragraph{}Le deuxième exercice avait pour but de nous faire modifier l'architecture afin que cette dernière soit capable de supporter des instructions s'exécutant sur plusieurs cycles, c'est-à-dire des instructions qui sont équivalentes à des enchaînements de plusieurs instructions.

A titre d'exemple, on peut parler de l'instruction X86 \verb+enter+ (décrite dans le projet) qui est équivalente à \verb+push %ebp+ suivi de \verb+rrmovl %esp,%ebp+.

\paragraph{}Enfin, le troisième exercice nous demandait d'ajouter plusieurs instructions à l'architecture, et notamment l'instruction \verb+enter+.

\paragraph{}Ce rapport est le compte-rendu du travail sur le projet que notre binôme, constitué de Paul Beziau et de Candice Bentéjac (groupe IN400A2), a effectué.



\section{Exercice 1}



\section{Exercice 2}



\section{Exercice 3}

\section*{Conclusion}

\end{document}
